\chapter{PENDAHULUAN}
\label{chap:pendahuluan}

% Ubah bagian-bagian berikut dengan isi dari pendahuluan

\section{Latar Belakang}
\label{sec:latarbelakang}

Tempat tinggal merupakan salah satu kebutuhan mendasar bagi setiap individu. 
Tempat tinggal tidak hanya berfungsi sebagai sarana fisik untuk perlindungan, tetapi juga berperan dalam pemenuhan aspek psikologis dan sosial individu. 
Mengingat pentingnya peran tempat tinggal dalam kehidupan, keputusan untuk memilih properti menjadi proses yang sangat krusial. 
Sebagian besar individu jarang melakukan transaksi pembelian atau penyewaan properti sepanjang hidup mereka, sehingga setiap keputusan terkait tempat tinggal membutuhkan pertimbangan yang matang dan menyeluruh. 
Kompleksitas ini tidak hanya muncul dari tingginya nilai ekonomi properti, tetapi juga karena banyaknya faktor yang perlu diperhatikan, seperti lokasi, harga, ukuran, fasilitas, serta preferensi pribadi \parencite{Gharahighehi2021}. 
Oleh karena itu, proses pemilihan tempat tinggal sering kali melibatkan pencarian informasi yang intensif, baik secara konvensional maupun melalui platform digital.

Di era digital saat ini, penggunaan platform daring untuk mencari properti mengalami peningkatan signifikan. 
Konsumen semakin beralih ke situs web dan aplikasi untuk mengeksplorasi pilihan properti yang tersedia. 
Namun, meskipun akses informasi menjadi lebih mudah, pengguna sering menghadapi tantangan baru dalam bentuk "\emph{information overload}" atau kelebihan informasi. 
Banyaknya pilihan properti yang tersedia justru dapat membuat proses pencarian menjadi melelahkan dan membingungkan. 
Pengguna perlu menyaring berbagai opsi yang tidak relevan sebelum menemukan properti yang sesuai dengan preferensi mereka. 
Hal ini menyebabkan pengalaman pencarian yang tidak optimal dan berpotensi menurunkan kepuasan pengguna \parencite{YuYonghongandWang2018}. 
Untuk mengatasi permasalahan ini, Sistem Rekomendasi (SR) menjadi salah satu solusi yang efektif. 
SR berfungsi untuk menyederhanakan proses pencarian dengan mengidentifikasi properti yang paling relevan bagi pengguna berdasarkan data yang diperoleh dari interaksi mereka dengan platform.

Berbagai pendekatan telah dikembangkan untuk meningkatkan performa sistem rekomendasi properti. 
Salah satu penelitian yang signifikan adalah karya Han Jong Jun et al., yang menghasilkan SR berbasis embedding bernama “SeoulHouse2Vec”. 
SR ini menggunakan pendekatan \emph{Neural Network Collaborative Model} yang memanfaatkan teknik embedding untuk merepresentasikan hubungan antara pengguna dan properti secara lebih baik \parencite{Jun2020}. 
Penelitian lain yang dilakukan oleh Zhang et al. mengusulkan pendekatan \emph{Content-Based Filtering} yang menggunakan model dua tahap, di mana sistem merekomendasikan properti berdasarkan karakteristik item serta preferensi historis pengguna \parencite{Zhang2019}. 
Terdapat penelitian lain tentang SR properti yang menggunakan \emph{Content-Based Filtering} dengan mengimplementasikan metode \emph{Term Frequency Inverse Document Frequency} (TF-IDF) untuk memberikan bobot pada judul, deskripsi, dan alamat sebuah iklan properti yang dikunjungi pengguna. 
Kemudian algoritma Apriori digunakan untuk memeberikan rekomendasi properti yang mirip dengan yang pengguna lihat \parencite{Badriyah2018}. 
Meskipun pendekatan ini menunjukkan hasil yang menjanjikan, masih terdapat beberapa tantangan yang perlu diatasi dalam pengembangan SR yang optimal, terutama dalam menangani berbagai keterbatasan data pengguna.

\section{Permasalahan}
\label{sec:permasalahan}

Dari permasalahan tersebut maka \lipsum[1][1-6]

\section{Tujuan}
\label{sec:Tujuan}

Tujuan dari \lipsum[1][1-3] adalah:

\begin{enumerate}[nolistsep]

  \item Membuat \lipsum[2][1-3]

  \item \lipsum[3][1-3]

\end{enumerate}

\section{Batasan Masalah}
\label{sec:batasanmasalah}

Batasan-batasan dari \lipsum[1][1-3] adalah:

\begin{enumerate}[nolistsep]

  \item Mempermudah \lipsum[2][1-3]

  \item \lipsum[3][1-5]

  \item \lipsum[4][1-5]

\end{enumerate}

\section{Sistematika Penulisan}
\label{sec:sistematikapenulisan}

Laporan penelitian tugas akhir ini terbagi menjadi \lipsum[1][1-3] yaitu:

\begin{enumerate}[nolistsep]

  \item \textbf{BAB I Pendahuluan}

        Bab ini berisi \lipsum[2][1-5]

        \vspace{2ex}

  \item \textbf{BAB II Tinjauan Pustaka}

        Bab ini berisi \lipsum[3][1-5]

        \vspace{2ex}

  \item \textbf{BAB III Desain dan Implementasi Sistem}

        Bab ini berisi \lipsum[4][1-5]

        \vspace{2ex}

  \item \textbf{BAB IV Pengujian dan Analisa}

        Bab ini berisi \lipsum[5][1-5]

        \vspace{2ex}

  \item \textbf{BAB V Penutup}

        Bab ini berisi \lipsum[6][1-5]

\end{enumerate}
