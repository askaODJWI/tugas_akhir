\chapter{PENDAHULUAN}
\label{chap:pendahuluan}

% Ubah bagian-bagian berikut dengan isi dari pendahuluan

\section{Latar Belakang}
\label{sec:latarbelakang}

Tempat tinggal merupakan salah satu kebutuhan mendasar bagi setiap individu. 
Tempat tinggal tidak hanya berfungsi sebagai sarana fisik untuk perlindungan, tetapi juga berperan dalam pemenuhan aspek psikologis dan sosial individu. 
Mengingat pentingnya peran tempat tinggal dalam kehidupan, keputusan untuk memilih properti menjadi proses yang sangat krusial. 
Sebagian besar individu jarang melakukan transaksi pembelian atau penyewaan properti sepanjang hidup mereka, sehingga setiap keputusan terkait tempat tinggal membutuhkan pertimbangan yang matang dan menyeluruh. 
Kompleksitas ini tidak hanya muncul dari tingginya nilai ekonomi properti, tetapi juga karena banyaknya faktor yang perlu diperhatikan, seperti lokasi, harga, ukuran, fasilitas, serta preferensi pribadi \parencite{Gharahighehi2021}. 
Oleh karena itu, proses pemilihan tempat tinggal sering kali melibatkan pencarian informasi yang intensif, baik secara konvensional maupun melalui platform digital.

Di era digital saat ini, penggunaan platform daring untuk mencari properti mengalami peningkatan signifikan. 
Konsumen semakin beralih ke situs web dan aplikasi untuk mengeksplorasi pilihan properti yang tersedia. 
Namun, meskipun akses informasi menjadi lebih mudah, pengguna sering menghadapi tantangan baru dalam bentuk "\emph{information overload}" atau kelebihan informasi. 
Banyaknya pilihan properti yang tersedia justru dapat membuat proses pencarian menjadi melelahkan dan membingungkan. 
Pengguna perlu menyaring berbagai opsi yang tidak relevan sebelum menemukan properti yang sesuai dengan preferensi mereka. 
Hal ini menyebabkan pengalaman pencarian yang tidak optimal dan berpotensi menurunkan kepuasan pengguna \parencite{YuYonghongandWang2018}. 
Untuk mengatasi permasalahan ini, Sistem Rekomendasi (SR) menjadi salah satu solusi yang efektif. 
SR berfungsi untuk menyederhanakan proses pencarian dengan mengidentifikasi properti yang paling relevan bagi pengguna berdasarkan data yang diperoleh dari interaksi mereka dengan platform.

Berbagai pendekatan telah dikembangkan untuk meningkatkan performa sistem rekomendasi properti. 
Salah satu penelitian yang signifikan adalah karya Han Jong Jun et al., yang menghasilkan SR berbasis embedding bernama “SeoulHouse2Vec”. 
SR ini menggunakan pendekatan \emph{Neural Network Collaborative Model} yang memanfaatkan teknik embedding untuk merepresentasikan hubungan antara pengguna dan properti secara lebih baik \parencite{Jun2020}. 
Penelitian lain yang dilakukan oleh Zhang et al. mengusulkan pendekatan \emph{Content-Based Filtering} yang menggunakan model dua tahap, di mana sistem merekomendasikan properti berdasarkan karakteristik item serta preferensi historis pengguna \parencite{Zhang2019}. 
Terdapat penelitian lain tentang SR properti yang menggunakan \emph{Content-Based Filtering} dengan mengimplementasikan metode \emph{Term Frequency Inverse Document Frequency} (TF-IDF) untuk memberikan bobot pada judul, deskripsi, dan alamat sebuah iklan properti yang dikunjungi pengguna. 
Kemudian algoritma Apriori digunakan untuk memeberikan rekomendasi properti yang mirip dengan yang pengguna lihat \parencite{Badriyah2018}. 
Meskipun pendekatan ini menunjukkan hasil yang menjanjikan, masih terdapat beberapa tantangan yang perlu diatasi dalam pengembangan SR yang optimal, terutama dalam menangani berbagai keterbatasan data pengguna.

Salah satu permasalahan utama yang dihadapi oleh SR dalam domain properti adalah fenomena \emph{cold-start problem}, di mana sistem kesulitan memberikan rekomendasi yang akurat bagi pengguna baru atau bagi pengguna yang memiliki sedikit riwayat interaksi dengan platform. 
Sebagian besar pendekatan saat ini, seperti \emph{Content-Based Filtering}, hanya berfokus pada karakteristik item atau aktivitas historis pengguna, seperti klik atau preferensi visual. 
Namun, model ini memiliki keterbatasan, karena perilaku pengguna yang terekam tidak selalu merepresentasikan preferensi nyata mereka. Pengguna mungkin melakukan klik secara acak atau terpengaruh oleh faktor eksternal, sehingga hasil rekomendasi yang diberikan bisa saja tidak sesuai dengan profil atau kebutuhan sebenarnya \parencite{KnollJulianandGro2018}. 
Oleh karena itu, diperlukan pendekatan yang lebih personal dan komprehensif dalam menangkap preferensi pengguna.

Untuk mengatasi tantangan tersebut, tugas akhir ini akan mengembangkan sistem rekomendasi berbasis \emph{Hybrid Filtering} yang menggabungkan \emph{Content-Based Filtering} dan \emph{Knowledge-Based Recommender Systems} dengan menggunakan metode \emph{Profile Matching} yang berfokus pada karakteristik demografi dan kondisi sosial pengguna. 
Pendekatan ini tidak hanya akan meningkatkan kualitas rekomendasi, tetapi juga dapat mengurangi risiko \emph{cold-start problem} yang sering terjadi pada SR konvensional. 
Tugas akhir ini diharapkan dapat memberikan kontribusi signifikan dalam pengembangan sistem rekomendasi properti yang lebih efektif dan efisien di era digitalisasi ini.

\section{Rumusan Masalah}
\label{sec:permasalahan}

Berdasarkan latar belakang permasalahan di atas, maka rumusan permasalahan dalam tugas akhir ini adalah sebagai berikut:

\begin{enumerate}

  \item Apa saja data demografi dan kondisi sosial yang relevan untuk membentuk profil pembeli?

  \item Bagaimana algoritma \emph{Profile Matching} dapat digunakan untuk mencocokkan properti dengan profil pembeli?

  \item Bagimana teknik \emph{Hybrid Filtering} dapat diterapkan untuk menentukan properti yang mirip berdasarkan karakteristik item?
  
\end{enumerate}

\section{Batasan Masalah}
\label{sec:batasanmasalah}

Berdasarkan rumusan masalah yang telah dibuat, maka dapat didefinisikan batasan-batasan dalam tugas akhir ini sebagai berikut:

\begin{enumerate}

  \item Tugas akhir ini akan fokus pada pengembangan sistem rekomendasi properti berbasis \emph{Hybrid Filtering} yang mengintegrasikan \emph{Content Based Filtering} dan \emph{Knowledge Based Recommender Systems} dengan metode \emph{Profile Matching}.

  \item Data yang digunakan dalam tugas akhir ini terbatas pada properti yang berada di daerah JABODETABEK dan Surabaya Kota yang diperoleh dari platform digital penjualan properti dengan metode \emph{scraping}.

  \item Demografi dan kondisi sosial pengguna yang digunakan untuk membangun profile pengguna diperoleh dari studi literatur penelitian terdahulu.
  
  \item Tugas akhir ini tidak akan mencakup analisis mendalam terhadap faktor eksternal yang mempengaruhi preferensi pengguna, seperti tren pasar properti atau perubahan ekonomi.

\end{enumerate}

\section{Tujuan}
\label{sec:Tujuan}

Berdasarkan rumusan masalah yang telah diuraikan sebelumnya, tujuan yang ingin dicapai pada tugas akhir ini adalah:

\begin{enumerate}

  \item Mengidentifikasi dan menentukan data demografi serta kondisi sosial yang relevan untuk membentuk profil pembeli properti.

  \item Mengembangkan dan menerapkan algoritma Profile Matching untuk mencocokkan properti dengan profil pembeli berdasarkan karakteristik pengguna.
  
  \item Menerapkan teknik \emph{Hybrid Filtering} untuk menentukan properti yang mirip berdasarkan karakteristik item, seperti lokasi, harga, ukuran, dan fasilitas.
  
\end{enumerate}

\section{Manfaat}
\label{sec:Manfaat}

Manfaat yang diharapkan dari tugas akhir ini adalah:

\begin{enumerate}

  \item Bagi Industri Properti, membantu dalam memahami preferensi konsumen dengan lebih baik, sehingga dapat menyesuaikan penawaran properti sesuai dengan kebutuhan pasar.

  \item Bagi pengguna platform pencarian properti, tugas akhir ini akan membantu mereka menemukan properti yang lebih relevan dan sesuai dengan profil dan kebutuhan spesifik mereka, sehingga proses pencarian menjadi lebih mudah dan efisien.

  \item Bagi pengembang sistem rekomendasi, tugas akhir ini dapat menjadi acuan dalam mengembangkan sistem rekomendasi yang lebih personal dan dapat mengatasi masalah \emph{cold-start}, terutama pada aplikasi pencarian properti.

  \item Bagi penulis, tugas akhir ini membantu menambah wawasan dalam hal sistem rekomendasi dan profile matching, serta penerapannya dalam dunia nyata.

\end{enumerate}

\section{Relevansi}
\label{sec:Relevansi}

Tugas akhir ini memiliki relevansi terhadap salah satu mata kuliah laboratorium Rekayasa Data dan Intelegensi Bisnis (RDIB), yaitu Pemodelan Sistem Kognitif. 
Pada mata kuliah tersebut, diajarkan teori-teori tentang sistem rekomendasi yang membantu dalam pembuatan tugas akhir ini. 
Salah satu \emph{roadmap} penelitian laboratorium RDIB, yaitu \emph{Recommender Systems}, memiliki kesesuaian dengan tugas akhir ini.

\begin{figure}[H]
  \centering

  % Ubah dengan nama file gambar dan ukuran yang akan digunakan
  \includegraphics[scale=0.1]{gambar/roadmap.png}

  % Ubah dengan keterangan gambar yang diinginkan
  \caption{\emph{Roadmap} Laboratorium RDIB.}
  \label{fig:roadmap}
\end{figure}